\chapter{The Hard Fork Combinator}
\label{hfc}

\section{Adjustments}

In this section we discuss some adjustments we had to make to the existing
design of the consensus layer to make the HFC possible

\subsection{Simplifying chain selection}
\label{simplifying-chain-selection}

Chain selection used to be given the full fragment; now it only gets the tip.

This is non-trivial; we have a long comment explaining this for
\lstinline!preferAnchoredCandidate!; we should move (copy?) that discussion
here, and also discuss this comment from \lstinline!preferCandidate!:

\begin{lstlisting}
  -- NOTE: An assumption that is quite deeply ingrained in the design of the
  -- consensus layer is that if a chain can be extended, it always should (e.g.,
  -- see the chain database spec in @ChainDB.md@). This means that any chain
  -- is always preferred over the empty chain, and 'preferCandidate' does not
  -- need (indeed, cannot) be called if our current chain is empty.
\end{lstlisting}


\subsection{Removing the assumption that slot/time conversion is always possible}
\label{removing-known-slot-assumption}

\section{Ledger}

\subsection{Invalid states}

\todo{This came from the Byron/Shelley appendix. Need to generalize a bit or provide context.}
In a way, it is somewhat strange to have the hard fork mechanism be part of the
Byron (\cref{byron:hardfork}) or Shelley ledger (\cref{shelley:hardfork})
itself, rather than some overarching ledger on top. For Byron, a Byron ledger
state where the \emph{major} version is the (predetermined) moment of the hard
fork is basically an invalid state, used only once to translate to a Shelley
ledger. Similar, the \emph{hard fork} part of the Shelley protocol version will
never increase during Shelley's lifetime; the moment it \emph{does} increase,
that Shelley state will be translated to the (initial) state of the post-Shelley
ledger.
