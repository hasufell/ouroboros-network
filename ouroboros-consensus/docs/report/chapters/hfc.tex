\chapter{The Hard Fork Combinator}
\label{hfc}

\section{Adjustments}

In this section we discuss some adjustments we had to make to the existing
design of the consensus layer to make the HFC possible

\subsection{Simplifying chain selection}
\label{simplifying-chain-selection}

Chain selection used to be given the full fragment; now it only gets the tip.

This is non-trivial; we have a long comment explaining this for
\lstinline!preferAnchoredCandidate!; we should move (copy?) that discussion
here, and also discuss this comment from \lstinline!preferCandidate!:

\begin{lstlisting}
  -- NOTE: An assumption that is quite deeply ingrained in the design of the
  -- consensus layer is that if a chain can be extended, it always should (e.g.,
  -- see the chain database spec in @ChainDB.md@). This means that any chain
  -- is always preferred over the empty chain, and 'preferCandidate' does not
  -- need (indeed, cannot) be called if our current chain is empty.
\end{lstlisting}


\subsection{Removing the assumption that slot/time conversion is always possible}
\label{removing-known-slot-assumption}

\section{Ledger}

\subsection{Invalid states}
\label{hfc:ledger:invalid-states}

\todo{This came from the Byron/Shelley appendix. Need to generalize a bit or provide context.}
In a way, it is somewhat strange to have the hard fork mechanism be part of the
Byron (\cref{byron:hardfork}) or Shelley ledger (\cref{shelley:hardfork})
itself, rather than some overarching ledger on top. For Byron, a Byron ledger
state where the \emph{major} version is the (predetermined) moment of the hard
fork is basically an invalid state, used only once to translate to a Shelley
ledger. Similar, the \emph{hard fork} part of the Shelley protocol version will
never increase during Shelley's lifetime; the moment it \emph{does} increase,
that Shelley state will be translated to the (initial) state of the post-Shelley
ledger.

\section{Failed attempts}

\subsection{Forecasting}

As part of the integration of any ledger in the consensus layer (not HFC
specific), we need a projection from the ledger \emph{state} to the consensus
protocol ledger \emph{view}
(\cref{class:ConsensusProtocol:ledgerview,ledger:api:LedgerSupportsProtocol}).
As we have seen\todo{Once we write these sections, add back references here},
the HFC requires for each pair of consecutive eras a pair of \emph{state} translation
functions, as well as a \emph{projection} from the state of the old era to the
ledger view of the new era. These means that if we have $n + 1$ eras, we need
$n$ across-era projection functions, in addition to the $n + 1$ projections
functions we already have \emph{within} each era.

This might feel a bit cumbersome; perhaps a more natural approach would be to
only have within-era projection functions, but require a function to translate
the ledger view (in addition to the ledger state) for each pair of eras.
We initially tried this approach; when projecting from an era to the next,
we would first ask the old era to give us the final ledger view in that era,
and then translate this final ledger view across the eras:

\begin{center}
\begin{tikzpicture}[
square/.style={rectangle, draw},
]
% old ledger
\node[square] (Astate) {old ledger state};
\node[square] (Aview1) [below=of Astate] {view};
\node[square] (Aview2) [right=of Aview1] {view};
\node         (Adots)  [right=of Aview2] {$\ldots$};
\node[square] (AviewN) [right=of Adots]  {view};
\draw[->] (Astate.south) -- (Aview1.north);
\draw[->] (Astate.south) .. controls +(down:1cm) and +(up:1cm).. (Aview2.north);
\draw[->] (Astate.south) .. controls +(down:1cm) and +(up:1cm).. (AviewN.north);
%
% some intermediate nodes for positiiong
\node (AstateN) [above=of AviewN] {};
\node (mid) [right=of AstateN] {};
\node (midH) [above=of mid] {era boundary};
\node (midM) [below=of mid] {};
\node (midL) [below=of midM] {};
%
% new ledger
\node[square] (Bstate) [right=of mid] {new ledger state};
\node[square] (Bview1) [below=of Bstate] {view};
\node[square] (Bview2) [right=of Bview1] {view};
\node         (Bdots)  [right=of Bview2] {$\ldots$};
\node[square] (BviewN) [right=of Bdots]  {view};
\draw[->] (Bstate.south) -- (Bview1.north);
\draw[->] (Bstate.south) .. controls +(down:1cm) and +(up:1cm).. (Bview2.north);
\draw[->] (Bstate.south) .. controls +(down:1cm) and +(up:1cm).. (BviewN.north);
%
\draw[dotted] (midH) -- (midL);
\draw[->, dashed] (AviewN.south) .. controls +(down:1cm) and +(down:1cm) .. (Bview2.south) node[pos=0.5, below] {\emph{translate}};;
\end{tikzpicture}
\end{center}

The problem with this approach is that the ledger view only contains a small
subset of the ledger state; the old ledger \emph{state} might contain
information about scheduled changes that should be taken into account when
constructing the ledger view in the new era, but the final ledger view in the
old era might not have that information.

Indeed, a moment's reflection reveals that this cannot be right the approach.
After all, we cannot step the ledger state; the dashed arrow in
%
\begin{center}
\begin{tikzpicture}[
square/.style={rectangle, draw},
]
\node[square] (state) {ledger state at anchor};
\node[square] (view1) [below=of state] {view};
\node[square] (view2) [right=of view1] {view};
\node         (dots)  [right=of view2] {$\ldots$};
\node[square] (viewN) [right=of dots]  {view};
\draw[->] (state.south) -- (view1.north);
\draw[->] (state.south) .. controls +(down:1cm) and +(up:1cm).. (view2.north);
\draw[->] (state.south) .. controls +(down:1cm) and +(up:1cm).. (viewN.north);
\draw[->, dashed] (view1.south) .. controls +(down:1cm) and +(down:1cm) .. (view2.south) node[pos=0.5, below] {\emph{(impossible)}};
\end{tikzpicture}
\end{center}
%
is not definable: scheduled changes are recorded in the ledger state, not in
the ledger view.

We cannot forecastly directly from the old ledger state to the new era either: this would result in a ledger view from the old era in the new era, violating the invariant we discussed in \cref{hfc:ledger:invalid-states}, and it would moreover result in incorrect forecast
bound checks.

Requiring a special forecasting function for each pair of eras of course in a
way is cheating: it pushes the complexity of doing this forecasting to the
specific ledgers that the HFC is instantiated at. However, as it turns out, this
function tends to be easy to define for any pair of concrete ledgers; it's just
hard to define in a completely general way.
